\documentclass[10pt,twocolumn]{article}

\usepackage{times}
\usepackage{mathptmx}  
\usepackage{spverbatim}

%% Text-kodning, språk samt PS-font
\usepackage[swedish]{babel}
\usepackage[utf8]{inputenc}
\usepackage[T1]{fontenc}
\usepackage{ae,aecompl}

% % bitmap-grafik
\usepackage{graphicx}
% % matematik
\usepackage{mathtools}
\usepackage{latexsym}
\usepackage{graphicx}

\raggedbottom
\sloppy

\title{Laborationsrapport i TSKS10 \emph{Signaler, Information och Kommunikation}}

\author{Alexander Poole \\ alepo020, 920829-0057 }

\date{\today}

\begin{document}

\maketitle

\section{Inledning}

Denna laboration gick ut på att demodulera en smalbandig signal som blivit I/Q-modulerad och passerat ett delvis okänt filter. Givet är signalen $y(t)$ som har blivit samplad i $400$ kHz, se (\ref{ekv:y(t)}).

Då laborationen har blivit korrekt löst ska $y(t)$ demoduleras och två ordspråk identifierats 

\begin{equation}
y(t)=x(t-\tau_1)+0.9x(t-\tau_2)
\label{ekv:y(t)}
\end{equation} 

Det är givet att $x(t)$ är en smalbandig signal som kan beskrivas med följande formel (\ref{ekv:x-signal}). Där $w(t)$ är en störsignal som innehåller två andra signaler som är I/Q-modulerade kring andra bärfrekvenser.

\begin{equation}
x(t)=x_I(t)\cos(2\pi f_c t)-x_Q(t)\sin(2\pi f_c t)+w(t)
\label{ekv:x-signal}
\end{equation}
%%%%%%%%%%%%%%%%%%%%%%%%%%%%%%%%%%%%%%%%%%%%%%%%%%%%%%%%%
\section{Metod}

Jag började med att försöka bestämma bärfrekvensen $f_c$, detta gjordes genom att Fouriertransformera $y(t)$ till $Y(f)$. Figur \ref{fig:yf} visar de tre olika smalbandiga signaler som $y(t)$ innehåller. Utifrån labhandledningen vet man att bärfrekvensen, $f_c$, för dessa signaler är en multipel av $19$ kHz. Från figur \ref{fig:yf} kan man avläsa att de olika signalerna har $f_c=38$ kHz, $f_c=95$ kHz och $f_c=133$ kHz. 

\begin{figure}
\centering
\includegraphics[width=0.4\textwidth]{bilder/y-signal.png}
\caption{$Y(f)$ Fouiertransformen av $y(t)$}
\label{fig:yf}
\end{figure}

\begin{figure}
\centering
\includegraphics[width=0.4\textwidth]{bilder/y-filtrerad.png}
\caption{$y(t)$s $3$ olika smalbandiga signaler.}
\label{fig:y-filtrerad}
\end{figure}

Från labhandledningen får man också fram att den smalbandiga signalen i $y(t)$ som innehåller $x(t)$ består av tre olika signaler, först musik, sedan tal och sist brus. Genom att köra $y(t)$ genom tre olika bandpassfilter kan vi isolera respektive smalbandiga signal kring deras olika bärfrekvenser. Från figur \ref{fig:y-filtrerad} ser vi att den enda signal som kan innehålla $x(t)$ är den med bärfrekvens $f_c = 38$ kHz.

\begin{equation}
\begin{split}
z(\tau) = \int_{-\infty}^\infty y(t)y(t+\tau)\,\mathrm{d}t=
\int_{-\infty}^\infty (x(t-\tau_1)\\x(t-\tau_1+\tau)+0,9x(t-\tau_2)x(t-\tau_1+\tau)\\
+ 0,9x(t-\tau_2)x(t-\tau_1+\tau) + \\0,81x(t-\tau_2)x(t-\tau_2+\tau)\,\mathrm{d}t
\end{split} 
\label{ekv:mangfald}
\end{equation} 

Nästa sak som ska bestämmas är $|\tau_1 - \tau_2|$. För att bestämma detta utnytjades att en tredjedel av signalen är vitt brus. Detta då vitt brus är helt okorrelerat. Tittar man på (\ref{ekv:mangfald}) ser man att det även bör finnas toppar vid  $|\tau_1-\tau_2|$ detta på grund av ekot i signalen. Detta bekräftas även genom att kolla på figur \ref{fig:corr} som har två toppar vid $\pm 0.41$ sekunder. Topparnas avstånd till centrum motsvarar ekots tidförskjutning d.v.s. $|\tau_1-\tau_2| = 0.41$ sekunder.

\begin{figure}
\centering
\includegraphics[width=0.4\textwidth]{bilder/correlerad.png}
\caption{Mångtydlighetsfunktionen av det vita bruset i signalen $y(t)$ runt  $f_c = 38$ kHz}
\label{fig:corr}
\end{figure}

Det är inte nödvändigt att behöva ta reda på $\tau_1$ och $\tau_2$ det intressanta för att kunna skapa ett inversfilter är filtrets fördröjning d.v.s $|\tau_1-\tau_2|$. Vet man $\tau_1-\tau_2$ så kan man skapa ett inversfilter för att ta bort ekot i $y(t)$. Först tas de sampler som finns innan ekot startade d.v.s. de första $0,41$ sekunderan av $y(t)$, sedan används (\ref{ekv:invfilt}) för att filtrera bort ekot och på så sätt få ut $x_n(t)$. $x_n(t)$ är inte samma som insignalen utan en tidsförskjuten variant av den, vi vet ju fortfarande inte värdet på $\tau_1$ och $\tau_2$. Små tidsförskjutningar motsvarar en fasförskjutning för smalbandiga signaler.

\begin{equation}
x_n(t)=y(t) - 0.9x(t-|\tau_1-\tau_2|)
\label{ekv:invfilt}
\end{equation} 

%\begin{figure}
%\centering
%\includegraphics[width=0.4\textwidth]{bilder/x-signal.png}
%\caption{Visar y(t) vid $f_c = 38$ kHz före och efter inversfiltrering.}
%\label{fig:x-signal}
%\end{figure}

Det som är kvar för att kunna höra de två ordspråken är att I/Q-demodulera $x_n(t)$ till $x_{nI}(t)$ och $x_{nQ}(t)$, de tas också och lågpassfiltreras för att bli av med högfrekvent brus, detta görs med (\ref{ekv:xi}).  
Med hjälp av fasförskjutningskonstanten $\phi$ kan vi motverka fasförskjutningen som uppståt från inversfiltret. $\phi$ prövas fram genom att testa värden på $\phi$ mellan $0$ och $\frac{\pi}{2}$. Värden större än $\frac{\pi}{2}$ är onödiga då $\sin$ blir $\cos$ och vise versa. För signalen i denna rapport gjorde $\phi = \frac{\pi}{6}$ att man kunde urskilja ordspråken. 

\begin{equation}
\begin{split}
x_{nI}(t)= \mathcal{H}^{LP} 2x_n(t)\cos(2\pi f_c t - \phi)\\
x_{nQ}(t)= \mathcal{H}^{LP} -2x_n(t)\sin(2\pi f_c t - \phi)
\end{split}
\label{ekv:xi}
\end{equation} 

\section{Resultat och slutsats}

Jag har funnit att:
\begin{itemize}
\item Bärfrekvensen var $f_c=38$ kHz
\item $|\tau_1 - \tau_2|$ var $0.41$ s
\item Signalen innehöll ordspråket "Man ska inte kasta sten i glashus" 
\item Signalen innehöll ordspråket "Liten tuva stjälper ofta stort lass"
\item Fasförskjutningen var $\phi= \frac{\pi}{6}$
\end{itemize}
\clearpage


\end{document}


